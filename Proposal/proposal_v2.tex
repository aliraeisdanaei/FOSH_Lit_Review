\documentclass[conference]{IEEEtran}
\IEEEoverridecommandlockouts
% The preceding line is only needed to identify funding in the first footnote. If that is unneeded, please comment it out.
\usepackage{cite}
\usepackage{amsmath,amssymb,amsfonts}
\usepackage{algorithmic}
\usepackage{graphicx}
\usepackage{textcomp}

\usepackage{xcolor}
\def\BibTeX{{\rm B\kern-.05em{\sc i\kern-.025em b}\kern-.08em
    T\kern-.1667em\lower.7ex\hbox{E}\kern-.125emX}}
\begin{document}

\title{Initial Project Description}

\author{\IEEEauthorblockN{Ali Raeisdanaei, Jingyue Zhang, Tiantian Lin , Ziqian Qiu}
\IEEEauthorblockA{
\textit{University of Toronto}\\
}
}

\maketitle

\section{Research Question Description}
The Free and Open-Source Software (FOSS) movement has made many strides over the years. Free and Open-Source Hardware (FOSH) is an extension of the ethos of FOSS. It aims to bring many of the free as in libre values to the design and making of hardware to ensure privacy and ethical values interests of users. FOSH has also had many improvements over the years, but it seems to have been slower than FOSH.
We propose to perform a systematic literature review of the research and industrial applications of FOSH to understand the following research questions:

RQ1: What is the state-of-the art FOSH and what are the performance of this best FOSH? How does it compare in terms of performance and other aspects with the non-FOSH state-of-the-art? 

RQ2: What are the main challenges and drawbacks associated with the development, adoption, and sustainability of FOSH?

RQ3: How does non-FOSH compare with FOSH in terms of design and licsensing?

RQ4: To what extent is FOSS more adapted FOSH development, and what are the factors that contribute to any observed differences in their adaptability?

RQ5: What are the potential future developments, opportunities, and challenges that FOSH is facing, and what are the implications of these for the growth and sustainability of the FOSH movement?

\section{Methodology}
The proposed research methodology comprises the literature review and expert opinions. For the literature review part, we are considering using a systematic literature review, this type of literature review can provide a summary of the topics of the literature and connect related themes to each other. We divide the whole literature review into three processes: 
1)To locate the relevant research papers about FOSH, we plan to use forward and backward snowballing to generate sources and use keyword searching to find more publications. 
2)The research will also be grounded in the work of the FOSS organization and its publications. This provides us with a rich background of research into this topic.
3)Next, we will aggregate the result along with the analysis of industrial and commercial FOSH. Similarly, forward and backward snowballing will also be used to find the industrial and commercial FOSH.
The above three approaches enable us to identify the most relevant resource from a huge amount of available information. 
Moreover, we will conduct semi-structured interviews with professionals specializing in open-source hardware. The professionals will provide valuable insight into FOSH and help us to find potential research gaps. 

\section{Expected Result and Contribution}

Combining the systematic literature review and expert opinions, our study is expected to yield several key results and contributions. First, the review is anticipated to identify a range of examples of commercial Free and Open Source Hardware (FOSH) and their performances, which can provide a comprehensive overview of the current state-of-the-art in this field. These findings can contribute to a deeper understanding of the successes and challenges faced by FOSH developers and the potential for future applications.

Second, the review is expected to compare the performance of the best FOSH products with regular hardware and identify the strengths and limitations of FOSH. This analysis can contribute to a better understanding of the unique features and value of FOSH, as well as identifying areas where further development is needed.

Third, the review is anticipated to identify the challenges that FOSH is facing and to provide recommendations for addressing these challenges. This can contribute to the development of more effective strategies for FOSH developers, architects, and researchers, as well as identifying new opportunities for collaboration and development.

Finally, the review is expected to predict future directions for FOSH, including its potential integration with emerging technologies, such as the Internet of Things (IoT) and artificial intelligence (AI), the development of more efficient and sustainable hardware, and the use of FOSH to support social and humanitarian causes. These predictions can contribute to a better understanding of the potential impact of FOSH on future technological and societal developments.

Overall, our research is not only expected to gather the current literature and industrial resources on the state of art Open Source Hardware (OSH), but also to create a framework that can organize, combine, and conceptualize the primary theoretical concepts and relationships that cover both open source and hardware elements\cite{b1}. By doing so, we aim to provide a conceptual framework that can identify gaps in research and development, pinpoint technology transfer requirements, and promote increased collaboration within the OSH community.


\begin{thebibliography}{00}
\bibitem{b1} Ackerman, John R. "Toward open source hardware." U. Dayton L. Rev. 34 (2008): 183.
\end{thebibliography}

\vspace{12pt}

\end{document}
