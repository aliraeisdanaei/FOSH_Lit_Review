\documentclass[final-report.tex]{subfiles}

\begin{document}

% Conclude the result we found and answer the research questions.
\subsection{RQA: What are the types, scopes, and applications in the FOSH community?}

\subsubsection{types} 
The FOSH community includes current popular open-source hardware platforms like Arduino and Raspberry Pi, as well as 3D printing technology and software. There are also electronic components. In addition, robotics automation technology, medical devices and educational resource platforms are also the major types of FOSH.

\subsubsection{scopes}
The FOSH community encompasses a wide range of scopes, from small-scale side projects to large-scale industrial projects. There are also research and development projects in academic and scientific settings. Some of the community-driven projects are also aimed at addressing social or environmental challenges.

\subsubsection{Application}
The FOSH community has a diverse set of applications, including the ones we mentioned above which are already divided into three categories. There are education at all levels from K-12 to higher education and vocational training, healthcare including medical devices and equipment, environmental monitoring and conservation, robotics and automation for industrial, agricultural, and domestic applications, scientific research and development including physics, chemistry, and biology, personal or community use small-scale project, and social and humanitarian projects aimed at addressing societal or environmental challenges.
\\\\
\subsection{RQB: }

\subsection{How do the collaborations of FOSH take place?}
Based on our results, there are many available platforms with different focuses for FOSH enthusiasts and practitioners to connect and collaborate online. These forums provide a space for sharing knowledge, troubleshooting technical issues, and collaborating on new projects.


\end{document}