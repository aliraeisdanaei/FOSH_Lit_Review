\documentclass[final-report.tex]{subfiles}

\begin{document}

% Conclude the result we found and answer the research questions.
\subsection{RQA: What are the types, scopes, and applications in the FOSH community?}

\subsubsection{Types} 
The FOSH community includes current popular open-source hardware platforms like Arduino and Raspberry Pi, as well as 3D printing technology and software. There are also electronic components. In addition, robotics automation technology, medical devices and educational resource platforms are also the major types of FOSH.

\subsubsection{Scopes}
The FOSH community encompasses a wide range of scopes, from small-scale side projects to large-scale industrial projects. There are also research and development projects in academic and scientific settings. Some of the community-driven projects are also aimed at addressing social or environmental challenges.

\subsubsection{Application}
The FOSH community has a diverse set of applications, including the ones we mentioned above which are already divided into three categories. 
There is education at all levels from K-12 to higher education and vocational training, healthcare including medical devices and equipment, environmental monitoring and conservation, robotics and automation for industrial, agricultural, and domestic applications, scientific research and development including physics, chemistry, and biology, personal or community use small-scale project, and social and humanitarian projects aimed at addressing societal or environmental challenges.

\subsection{Licences}
The CERN 

\subsection{How do the collaborations of FOSH take place?}
Based on our results, there are many available platforms with different focuses for FOSH enthusiasts and practitioners to connect and collaborate online. These forums provide a space for sharing knowledge, troubleshooting technical issues, and collaborating on new projects.

In addressing our third research question, "How do the collaborations of FOSH take place?", we have analyzed the literature and identified several key factors that contribute to successful collaboration in the realm of open-source hardware:
\begin{itemize}
    \item Innovation Climate: Our analysis revealed that a conducive environment is crucial for fostering collaboration in open-source hardware. Approximately 82\% of collaborations were found to occur in makerspaces, such as campus organizations and innovation engineering communities. The atmosphere of mutual respect and effective communication tools within these spaces encourages collaborative efforts.
    \item Benefit Motivation: The majority of open-source hardware collaborations are driven by a clear understanding of the potential benefits. Collaborations often form to reduce costs, for strategic and economic reasons, or to leverage the diverse motivations of community members. Voluntariness is a core concept in these collaborations, as communities cannot be artificially created or maintained.
    \item Corporate Involvement: Our findings indicate that 63\% of research results in open-source hardware collaborations are produced with the involvement of companies. We speculate that this is due to the ability of companies to facilitate the production of deliverables, guarantee security, and provide marketing, production, and sales resources. These factors contribute to the growth of open-source hardware communities and the proliferation of their work.
    \item Guided by Clear Hardware Development Principles: Successful open-source hardware collaborations are typically guided by well-defined hardware development principles, which help to ensure consistency and quality in the projects undertaken by the community.
\end{itemize}



\end{document}