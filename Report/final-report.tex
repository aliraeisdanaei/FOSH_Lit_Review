\documentclass[acmtog]{acmart}

\usepackage{hyperref}
\usepackage{subfiles}
\usepackage{natbib}
\usepackage{listings}
\lstset{
  basicstyle=\ttfamily,
  frame=single,
  columns=fullflexible,
  keepspaces=true,
  breaklines=true,
  numbers=left, % Add line numbers on the left
  numberstyle=\tiny, % Set the style for line numbers
  stepnumber=1, % Show line numbers for every line
  numbersep=5pt % Set the distance between line numbers and code text
}

\AtBeginDocument{%
  \providecommand\BibTeX{{%
    \normalfont B\kern-0.5em{\scshape i\kern-0.25em b}\kern-0.8em\TeX}}}
% \citestyle{acmauthoryear}

\begin{document}

\title{Final Report: FOSH Literature Review}

\author{Ali Raeisdanaei}
\affiliation{%
  \institution{University of Toronto}
  \country{Canada}
}

\author{Jingyue Zhang}
\affiliation{%
  \institution{University of Toronto}
  \country{Canada}
}

\author{Tiantian Lin}
\affiliation{%
 \institution{University of Toronto}
 \country{Canada}}

\author{Ziqian Qiu}
\affiliation{%
  \institution{University of Toronto}
  \country{Canada}}

\begin{abstract}
Our project was a systematic literature review of Free and Open Source Hardware (FOSH).
Since starting to look at the literature on this subject, we have learned many things. 

Firstly, the field is relatively new, yet somewhat vast at the same time. 
The types of hardware we are considering were very limited.
There have been two journals that have been started since 2017, and our project will base most of its review. 
This is good news for our project since it means our review is a systematic review of almost \textit{all} the literature on this subject. 

Given the new information, we have refactored and refined some of our research questions. 
Some questions from the proposal may be beyond the scope of a single paper to be answered, so some may be omitted altogether. 

You can see a repository of our project along with a working document \ref{sec:C} that goes over the details
\href{https://github.com/aliraeisdanaei/FOSH_Lit_Review/}{here}
(Not finished).
\end{abstract}

\keywords{Open source design, Open source hardware}


%%
%% This command processes the author and affiliation and title
%% information and builds the first part of the formatted document.
\maketitle

\section{Introduction}
\label{introduction}
\subfile{introduction}

% \section{Background Literature}
% We will include some of the analyses that previous researchers have done on this topic. The major references are: TODO

% \subsection{Where is the Freedom?}
% We will follow the definition of Stallman\cite{b0_stallman}, as well as expanding our research to define the "open source" in hardware
% \subsection{Where is the Hardware?}
% An overview of hardware and its difficulties is needed as good background information. 

\section{Methodology}
\label{methodology}
\subfile{methodology}


\section{Results}
The comprehensive results of our study can be accessed via the subsequent public repository:
\par
\url{https://github.com/aliraeisdanaei/FOSH_Lit_Review}

% TODO: FORMAT for Research Question A
FOSH is a new way of designing and building hardware that's become popular in many areas since it is open for innovation and can be democratically beneficial to the community. Based on the result of the systematic literature review, we can divide the perspective of FOSH in the following categories. electronics and computing, robotics and automation, and education and research.
\subsection{education and research}

% # aspect of education: electronic[4], auto vehicle design, industrial design, computer science, digital media design[3], K-12 student[92], control engineering education[98], robot[94], STEM education[]
% #kind of tool used: RISC-V based[18], Arduino[75],open-source openAirInterface Testbed project[62], Pantograph,3D printer, Hapkit[92], open-source robot platform(OSR)[94]
% #learning purpose, course: learning microprocessor, hardware based simulator[4], intergrate computer lab[18], microcontroller lab class[75], mobile communication network[62], robotic design[94], embedded system[]
% #newly invented devices for student to use: open-source power electronics didactic platform[61], open-source "ball-on-beam didactic device", attached to Arduino microcontroller [98], open-source robotic manipulation platform[148], development the front-end for the FPGA based platform[]
% #low-cost part: [3],[98], [148]


\subsection{electronics and computing}
Some of the most popular and successful hardware are open sourced, such as Risc-V and Arduino. These are free for modification, which is also commonly introduced and extended too various new products. For example, Mlakić1 et al. proposed a measurement device implemented based on Arduino. 
% TODO: REF 


\subsection{robotics and automation}
In general, robots are often created to either complete tasks that human are unable to, or to increase the productivity. VIKTOR III is a open source robot to improve the farming production quality by not only "empower[ing] individuals to grow their own food [but also] serve[ing] as platform for robotics education." This also falls into the category of the benefit of open source product can bring to the field of education. By improving on the current older invention, a open-sourced robot named FarmBot, VIKTOR III can be build only using 1/6 of its cost but with completely same feature. It also uses the state-of-art machine learning technology deep learning that provides a higher plant detection accuracy. The inventors also suggested that it can help astronouts harvest in space. By doing so, it provides a blueprint for future design and study purposes. There are many produces that share the same goal, such as the Yale OpenHand Project, which focus on improving the design process and produce various options for the researchers to adopt on.  The inventor of WoodenHaptics also agrees with the importance of innovation, which is why they published the blueprint for this design in order to "Lowering the barriers to inspire experimentation". As a realm that is constantly been developed and updated by new technologies, they believe the ultimate future goal for any open source robot is "identifying willing end users who will put their own design modifications online, thereby allowing progress in the research community to move even faster". In fact, this concept of republishing the modified product is also widley and strongly agreed by many researchers, which leads to the invention of ROS (Robot Operating System) that allows people to communicate. The aspect of FOSH communities will be further discussed later. 




\section{Discussion}
Based on the result, Discuss the potential future developments, opportunities, and challenges that FOSH is facing as well as identify the fields where we could focus more attention on FOSH.

% \section{Critique And Solutions}

\section{Conclusion}
Conclude the result we found and answer the research questions.


%%
%% The acknowledgments section is defined using the "acks" environment
%% (and NOT an unnumbered section). This ensures the proper
%% identification of the section in the article metadata, and the
%% consistent spelling of the heading.
\begin{acks}
to be finished
\end{acks}

%%
%% The next two lines define the bibliography style to be used, and
%% the bibliography file.
\bibliographystyle{ACM-Reference-Format}
\bibliography{ref}


%%
%% If your work has an appendix, this is the place to put it.
% \appendix

% \section{ToDo List}
% \begin{enumerate}
% \item Divide up the readings between all four members. (There are two journals since 2017 that should be read. 
% There are other commercial applications and companies selling FOSH that need to be considered. Other papers that have been found through the backward propagation method should also be added to the reading list.)
% \item Do the reading, design a literature map, and summarise briefly about each paper read
% \item Collecting information on the hardware found in terms of category, benchmarks, and licence. 
% \item Find benchmarks to compare with the non-FOSH
% \item Perform the analysis and write the report.
% \section{Working On Document}
% \label{sec:C}
% \end{enumerate}


\end{document}
\endinput
