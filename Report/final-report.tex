\documentclass[acmtog]{acmart}

\usepackage{hyperref}
\usepackage{subfiles}
\usepackage{natbib}
\usepackage{booktabs}
\usepackage{tabularx}
\usepackage{listings}

\lstset{
  basicstyle=\ttfamily,
  frame=single,
  columns=fullflexible,
  keepspaces=true,
  breaklines=true,
  numbers=left, % Add line numbers on the left
  numberstyle=\tiny, % Set the style for line numbers
  stepnumber=1, % Show line numbers for every line
  numbersep=5pt % Set the distance between line numbers and code text
}

\AtBeginDocument{%
  \providecommand\BibTeX{{%
    \normalfont B\kern-0.5em{\scshape i\kern-0.25em b}\kern-0.8em\TeX}}}
% \citestyle{acmauthoryear}

\begin{document}

\title{Final Report: FOSH Literature Review}

\author{Ali Raeisdanaei}
\affiliation{%
  \institution{University of Toronto}
  \country{Canada}
}

\author{Jingyue Zhang}
\affiliation{%
  \institution{University of Toronto}
  \country{Canada}
}

\author{Tiantian Lin}
\affiliation{%
 \institution{University of Toronto}
 \country{Canada}}

\author{Ziqian Qiu}
\affiliation{%
  \institution{University of Toronto}
  \country{Canada}}

\begin{abstract}
Our project was a systematic literature review of Free and Open Source Hardware (FOSH).
Since starting to look at the literature on this subject, we have learned many things. 

Firstly, the field is relatively new, yet somewhat vast at the same time. 
The types of hardware we are considering were very limited.
There have been two journals that have been started since 2017, and our project will base most of its review. 
This is good news for our project since it means our review is a systematic review of almost \textit{all} the literature on this subject. 

Given the new information, we have refactored and refined some of our research questions. 
Some questions from the proposal may be beyond the scope of a single paper to be answered, so some may be omitted altogether. 

You can see a repository of our project along with a working document \ref{sec:C} that goes over the details
\href{https://github.com/aliraeisdanaei/FOSH_Lit_Review/}{here}
(Not finished).
\end{abstract}

\keywords{Open source design, Open source hardware}

\maketitle

\section{Introduction}
\label{introduction}
\subfile{introduction}

\section{Methodology}
\label{methodology}
\subfile{methodology}

\section{Results}
\label{results}
\subfile{results}

% \section{Critique And Solutions}

\section{Conclusion}
\label{conclusion}
\subfile{conclusion}

\section{Discussion}
\label{discussion}
\subfile{discussion}

\begin{acks}
Our heartfelt thanks go to all our teammates, who have contributed their time and effort to this project. 
We would also like to express our appreciation and gratitude to Professor Shurui for her invaluable guidance and support throughout this research project. 
We would also extend our gratitude to the authors of all paper we reviewed, for their diligent work and commitment to producing a high-quality research publication.
\end{acks}

\bibliographystyle{ACM-Reference-Format}
\bibliography{ref}


%% If your work has an appendix, this is the place to put it.
% \appendix

\end{document}
\endinput
