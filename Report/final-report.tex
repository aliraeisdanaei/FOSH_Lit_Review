\documentclass[acmtog]{acmart}

\usepackage{hyperref}
\usepackage{subfiles}
\usepackage{natbib}
\usepackage{listings}
\lstset{
  basicstyle=\ttfamily,
  frame=single,
  columns=fullflexible,
  keepspaces=true,
  breaklines=true,
  numbers=left, % Add line numbers on the left
  numberstyle=\tiny, % Set the style for line numbers
  stepnumber=1, % Show line numbers for every line
  numbersep=5pt % Set the distance between line numbers and code text
}

\AtBeginDocument{%
  \providecommand\BibTeX{{%
    \normalfont B\kern-0.5em{\scshape i\kern-0.25em b}\kern-0.8em\TeX}}}
% \citestyle{acmauthoryear}

\begin{document}

\title{Final Report: FOSH Literature Review}

\author{Ali Raeisdanaei}
\affiliation{%
  \institution{University of Toronto}
  \country{Canada}
}

\author{Jingyue Zhang}
\affiliation{%
  \institution{University of Toronto}
  \country{Canada}
}

\author{Tiantian Lin}
\affiliation{%
 \institution{University of Toronto}
 \country{Canada}}

\author{Ziqian Qiu}
\affiliation{%
  \institution{University of Toronto}
  \country{Canada}}

\begin{abstract}
Our project was a systematic literature review of Free and Open Source Hardware (FOSH).
Since starting to look at the literature on this subject, we have learned many things. 

Firstly, the field is relatively new, yet somewhat vast at the same time. 
The types of hardware we are considering were very limited.
There have been two journals that have been started since 2017, and our project will base most of its review. 
This is good news for our project since it means our review is a systematic review of almost \textit{all} the literature on this subject. 

Given the new information, we have refactored and refined some of our research questions. 
Some questions from the proposal may be beyond the scope of a single paper to be answered, so some may be omitted altogether. 

You can see a repository of our project along with a working document \ref{sec:C} that goes over the details
\href{https://github.com/aliraeisdanaei/FOSH_Lit_Review/}{here}
(Not finished).
\end{abstract}

\keywords{Open source design, Open source hardware}

\maketitle

\section{Introduction}
\label{introduction}
\subfile{introduction}

% \section{Background Literature}
% We will include some of the analyses that previous researchers have done on this topic. The major references are: TODO

% \subsection{Where is the Freedom?}
% We will follow the definition of Stallman\cite{b0_stallman}, as well as expanding our research to define the "open source" in hardware
% \subsection{Where is the Hardware?}
% An overview of hardware and its difficulties is needed as good background information. 

\section{Methodology}
\label{methodology}
\subfile{methodology}

\section{Results}
<<<<<<< HEAD
The comprehensive results of our study can be accessed via the subsequent public repository:
\par
\url{https://github.com/aliraeisdanaei/FOSH_Lit_Review}

% TODO: FORMAT for Research Question A
FOSH is a new way of designing and building hardware that's become popular in many areas since it is open for innovation and can be democratically beneficial to the community. Based on the result of the systematic literature review, we can divide the perspective of FOSH in the following categories. electronics and computing, robotics and automation, and education and research.
\subsection{education and research}

% aspect of education: electronic[4], industrial design, computer science, digital media design[3], K-12 student[92], control engineering education[98], robot[94], autonomous car(robot)[232]

%kind of tool used: RISC-V based[18], Arduino[75],open-source openAirInterface Testbed project[62], Pantograph,3D printer, Hapkit[92], open-source robot platform(OSR)[94], GNURadio-Based Toolkit[300]

%learning purpose, course: learning microprocessor, hardware based simulator[4], intergrate computer lab[18], microcontroller lab class[75], mobile communication network[62], robotic design[94], embedded system[211], optical wireless communication[300], chip design[304]

%newly invented devices for student to use: open-source power electronics didactic platform[61], open-source "ball-on-beam didactic device", attached to Arduino microcontroller [98], open-source robotic manipulation platform[148], development the front-end for the FPGA based platform[211]

The proliferation of open-source hardware has led to its increasing adoption in educational institutions, ranging from K-12 to graduate programs.[92] This trend can be attributed to the wide applicability of open-source hardware in diverse fields such as electronics[4], computer science, digital media design[3], robotics,[94] and automated vehicles[232]. Taking open-source hardware into courses can provide students with a more comprehensive understanding of the relevant concepts and facilitate hands-on learning experiences.[3]

Arduino, and Arduino-based platforms, are the most prevalent open-source hardware utilized in schools. These tools enable the design of laboratory exercises that target the acquisition of knowledge and skills related to microcontrollers[75], microprocessors[4], embedded systems[211], mobile communication networks[62], optical wireless communications[300], and chip design[304]. The versatility and accessibility of open-source hardware make them ideal for fostering a deeper understanding of the basic principles.

What's more, the appearance of new open-source hardware platforms such as the power electronics didactic platform[61], the "ball-on-beam didactic device"[98], the robotic manipulation platform[148], and the development of the front-end for the FPGA-based platform[211] has further facilitated the teaching and learning of the hardware. In addition to the cost-saving benefits, these open-source hardware tools aim to inspire students to contribute to the existing open-source hardware ecosystem. As highlighted by A. J. Miller et al., the high cost of scientific tools will impede the pace of technological progress. Thus, the incorporation of open-source hardware in education can have benefits both in reducing cost and activating the development of the hardware.[75]


\subsection{electronics and healthcare}
Some of the most popular and successful hardware are open-sourced, such as Risc-V and Arduino. These are free for modification, which is also commonly introduced and extended too various new products. For example, Mlakić1 et al. proposed a measurement device implemented based on Arduino. This can be further used to making own IED capable for IoT and smart grid symbiosis. 
% TODO: REF FOR Mlakić1
Arduino can also be combined with various products to realize human-computer interaction. One example would be Yuning Fan's programming language-based interactive device, which is an interface designed not only for the potential benefit technologically, but also educate children at the same time. As mentioned in the previous paragraph, education is a major field that FOSH is contributing to an strongly impacted in today's community.  
% TODO: REF FOR  Yuning Fan
One major reason for any implementation of electronic or computing products is to reduce the cost of the current solution. Noted the vital and complex role of medical realm, some researches  focus on different fields of this branch. For example, the ultrasound test bench demonstrated by Pashaei et al. that can further extended to other devices. The key sellout of this product is that it's always in demand of medical equipment, where making the design open-sourced would greatly benefit the whole medical community. 
% TODO: REF for Pashaei
In addition, given the limitation of availability of affordable medical equipment to treat chronic condition, Dorin et al. explores the Loop Open-source Artificial Pancreas (APS).
% TODO: REF FOR Dorin
There are similar designs add convenience for health related measurement device such as OpenSenseRT which is also low cost and extendable,
% TODO: REF for OpenSenseRT
and the robotic platform to enable dexterous procedures within CT scanners which is practically a robot hand that allows physicians to localize tumours quickly. Robots are another major application area in the FOSH community, which will be further explore in the next paragraph. 



\subsection{robotics and automation}
In general, robots are often created to either complete tasks that human are unable to, or to increase the productivity. VIKTOR III is a open source robot to improve the farming production quality by not only "empower[ing] individuals to grow their own food [but also] serve[ing] as platform for robotics education." This also falls into the category of the benefit of open source product can bring to the field of education. By improving on the current older invention, a open-sourced robot named FarmBot, VIKTOR III can be build only using 1/6 of its cost but with completely same feature. It also uses the state-of-art machine learning technology deep learning that provides a higher plant detection accuracy. The inventors also suggested that it can help astronouts harvest in space. By doing so, it provides a blueprint for future design and study purposes. There are many produces that share the same goal, such as the Yale OpenHand Project, which focus on improving the design process and produce various options for the researchers to adopt on.  The inventor of WoodenHaptics also agrees with the importance of innovation, which is why they published the blueprint for this design in order to "Lowering the barriers to inspire experimentation". As a realm that is constantly been developed and updated by new technologies, they believe the ultimate future goal for any open-source robot is "identifying willing end users who will put their own design modifications online, thereby allowing progress in the research community to move even faster". In fact, this concept of republishing the modified product is also widley and strongly agreed by many researchers, which leads to the invention of ROS (Robot Operating System) that allows people to communicate. The aspect of FOSH communities will be further discussed later. 



=======
\label{results}
\subfile{results}
>>>>>>> main

\section{Discussion}
\label{discussion}
\subfile{discussion}

% \section{Critique And Solutions}

\section{Conclusion}
\label{conclusion}
\subfile{conclusion}

%%
%% The acknowledgments section is defined using the "acks" environment
%% (and NOT an unnumbered section). This ensures the proper
%% identification of the section in the article metadata, and the
%% consistent spelling of the heading.
\begin{acks}
to be finished
\end{acks}

%%
%% The next two lines define the bibliography style to be used, and
%% the bibliography file.
\bibliographystyle{ACM-Reference-Format}
\bibliography{ref}


%% If your work has an appendix, this is the place to put it.
% \appendix


\end{document}
\endinput
