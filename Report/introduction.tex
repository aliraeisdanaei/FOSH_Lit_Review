\documentclass[final-report.tex]{subfiles}
% \graphicspath{{../Images}}
% \addbibresource{ref.bib}

\begin{document}

\subsection{Background and Motivation}
The free and open source movement is defined by the following four principles:
\begin{quote}
    A program is a free software if the program’s users have the four essential freedoms:

    \begin{itemize}
        \item The freedom to run the program as you wish, for any purpose (freedom 0).
        \item The freedom to study how the program works, and change it so it does your computing as you wish (freedom 1). 
            Access to the source code is a precondition for this.
        \item The freedom to redistribute copies to help your neighbour (freedom 2).
        \item The freedom to distribute copies of your modified versions to others (freedom 3). 
            Doing this gives the whole community a chance to benefit from your changes. Access to the source code is a precondition for this.
    \end{itemize}
    A program is a free software if it gives users adequately all of these freedoms. Otherwise, it is non-free. 
    While we can distinguish various non-free distribution schemes in terms of how far they fall short of being free, we consider them all equally unethical.
\end{quote} \cite{b0_stallman}.

These four principles have started a movement that has reveloutinised technonlogy


\subsection{Research question}
\subsection{Scope and limitations}
\subsection{Organization of the paper}
\textbf{Then, we also need to explain why a literature review is needed.}
\textbf{What exactly does our work add?}


\end{document}