\documentclass[final-report.tex]{subfiles}
% \graphicspath{{../Images}}
% \addbibresource{ref.bib}

\begin{document}

\subsection{Background and Motivation}
The free and open source movement is defined by the following four principles:
\begin{quote}
    A program is a free software if the program’s users have the four essential freedoms:

    \begin{itemize}
        \item The freedom to run the program as you wish, for any purpose (freedom 0).
        \item The freedom to study how the program works, and change it so it does your computing as you wish (freedom 1). 
            Access to the source code is a precondition for this.
        \item The freedom to redistribute copies to help your neighbour (freedom 2).
        \item The freedom to distribute copies of your modified versions to others (freedom 3). 
            Doing this gives the whole community a chance to benefit from your changes. Access to the source code is a precondition for this.
    \end{itemize}
    A program is a free software if it gives users adequately all of these freedoms. Otherwise, it is non-free. 
    While we can distinguish various non-free distribution schemes in terms of how far they fall short of being free, we consider them all equally unethical.
\end{quote} \cite{b0_stallman}.

These four principles have started a movement that has revolutionised intellectual property and specifically technology. 
We have seen the vast social and technical benefits of this movement.
The free and open source movement has democratised access to information and technology: 
any person has access to the basic principles of \cite{b0_stallman} on the best software.
The movement's technical benefit has been the increase of innovation and collaboration. 
It is no wonder when neighbours help each other build, we have software like the Linux Kernel, Mozzila Firefox, and many others. 

These four principles have been applied to other fields than software. 
Similar to the free and open source software (FOSS), free and open source hardware (FOSH) is any piece of information that is needed to exercise the four principles as it applies to hardware. 
These include anything such as design files, blueprints, specifications, documentation, and even software for the building, designing, modifying, distributing, and using hardware. 
A common example of FOSH is \href[]{https://www.arduino.cc/}{Arduino} used for single board microcontrollers in a variety of applications. 

FOSH has been growing in interest over the years. 
This is evident in terms of the increasing number of projects, associations, literature. 
The Open Hardware Association tracks 
2057 
FOSH projects to date \cite{OSH_association_def}.
It also lists multiple journals that have started since 2017.
These include, 
\begin{itemize}
    \item \href{https://openhardware.metajnl.com/}{Journal of Open Hardware}
    \item \href{https://www.sciencedirect.com/journal/hardwarex}{HardwareX}
    \item \href{https://www.tjoe.org/}{The Journal of Open Engineering}
    \item Computers, Design and Technologies from MDPI
        \begin{itemize}
            \item \href{Computers}{https://www.mdpi.com/journal/computers}
            \item \href{Designs}{https://www.mdpi.com/journal/designs}
            \item \href{Technologies}{https://www.mdpi.com/journal/technologies}
        \end{itemize}
\end{itemize}

The new development FOSH provides researchers an interesting study on the open source movement outside the common FOSH movement. 
This is the goal of this project. 

\subsection{Related work}
Systematic reviews aim to address a series of research questions by identifying and elucidating existing knowledge gaps, contrasting hypotheses, or broadening the scope of subject matter within a specific area of expertise \cite{gough2017introduction}. 
The insights gained from systematic reviews enable stakeholders, practitioners, and researchers to make informed decisions and strategize future investigations to bridge identified gaps based on the accumulated evidence. 
Petticrew and Roberts \cite{petticrew2008systematic} assert that the initial phase in the development of a systematic literature review (SLR) involves determining the necessity of conducting a review on a particular topic.

Annually, approximately 2.5 million new scientific papers are published, necessitating secondary studies that synthesize and systematically organize the knowledge within a specific area. 
These studies benefit researchers by identifying research gaps and aiding practitioners in understanding the effectiveness of specific methods or technologies.

To the best of our knowledge, this paper is the sole secondary study overviewing the state of art free and open-source hardware(FOSH) . Other related studies include: 
\begin{itemize}
    \renewcommand{\labelitemi}{}
    \item 1) Saari et al.\cite{7973568}, surveyed multiple network sensor solutions utilizing Raspberry Pi for the Internet of Things; 
    \item 2) Sullivan and Heffernan \cite{sullivan2016robotic}, who conducted a systematic literature review on robotics construction kits in STEM disciplines
    \item 3) Heradio et al. \cite{heradio2018open} performed a systematic mapping study of OSHW in education, albeit dated; 
    \item 4) Ariza and Pearce \cite{ariza2022low} explored low-cost assistive technologies for disabled individuals using OSHW and software.
\end{itemize}
 However, most of these reviews primarily summarize the stages of hardware development and lack exploration into aspects such as licences, community collaboration, and related literature reviews.
 
\subsection{Research question}

To comprehend the complexity and diversity of Free and Open Source Hardware (FOSH), it is crucial to explore its various forms, scopes, and applications. Hence, we pose the following research question:

    \textbf{RQA: What are the types, scopes, and applications of FOSH?}
    
Rationale: Understanding the types, scopes, and applications of FOSH will enable researchers, practitioners, and stakeholders to better grasp the potential of FOSH in various domains and enhance its utilization and impact.
Answers to the application of FOSH can determine the success or limitations of the movement.

    \label{RQA}
    
Secondly, to measure the level of freedom as described by the four principles of \cite{b0_stallman}, we ask

    \textbf{RQB: What are the licences of the components of FOSH in each type?}
    
Rationale: Identifying the licenses of FOSH components will help in determining the level of freedom and openness, thus ensuring the compatibility and accessibility of the components across different FOSH projects.
The information on licencing could provide answers to the importance practitioners have on the principles, as well as how they are packaging their products. 

    \label{RQB}

Thirdly, we would like to understand the collaboration environments of FOSH.

    \textbf{RQC: How do the collaborations of FOSH take place?}
    
Rationale: Exploring the collaboration environments of FOSH projects will provide insights into the interaction dynamics between contributors, fostering better understanding and improvement of communication and cooperation strategies within the FOSH community.
Understanding collaboration environments is essential to understanding the social aspect of how people exercise the four principles of freedom defined by \cite{b0_stallman}.

    \label{RQC}
\subsection{Significance}
The results of this study are significant as they provide insights to understanding the free and open source movement. 
They can be used to improve aspects of FOSS or other similar movements both socially and technically. 


% \subsection{Scope and limitations}
% \subsection{Organization of the paper}


\end{document}