\subsection{literature review}
Our systematic literature review (SLR) adhered to the methodological guidelines and procedural steps established by Gough et al. [25] and Petticrew & Roberts [27], which can be encapsulated as follows:
\begin{itemize}
    \item 1) Formulation of research questions and a conceptual framework; 
    \item 2) Searching and screening for pertinent literature based on predetermined eligibility criteria
    \item 3) Coding the literature to align with the conceptual framework
    \item 4) Employing quality appraisal criteria;
    \item 5) Synthesizing the selected studies within the context of the conceptual framework or using study codes;
    \item 6) Interpreting and disseminating the findings. 
\end{itemize}
These steps were executed in conjunction with the PRISMA guidelines provided by [24] to facilitate the various phases of the review process. 
Figure 1 illustrates the stages of the SLR in accordance with the aforementioned guidelines.
To answer the question regarding the collaborative environments, our method is a grey literature review. 
The qualitative questions posed can be effectively answered by a systematic literature review, as the movement is very young. 
Most if not all the literature on the subject can be reviewed, along with all the projects listed by the Open Source Hardware Association.

% We started with a seed of papers on this subject, and we checked the citations used in the seeds recursively. 

% Part of the methodology would also be to read through all the literature in the two FOSH journals and to record summaries, benchmarks, and the licences of the hardware they proposed.
% These two journals are the 
% \begin{enumerate}
%     \item \href{https://openhardware.metajnl.com/}{Journal of Open Hardware}
%     \item \href{https://www.sciencedirect.com/journal/hardwarex}{HardwareX}
% \end{enumerate}
\subsection{Search Strategy and Selection Criteria}
\subsubsection{Search Scope}

The publications of the FOSH journals were the initial pool of publications for our search. 

\subsubsection{Screening And Selection}

\subsubsection{Snowballing}

\subsection{Data Extraction and Analysis}


\subsection{Limitations}