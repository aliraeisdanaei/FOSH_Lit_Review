
The main methodology of the systematic literature review is backward propagation. 
To answer the question regarding the collaborative environments, our method is a grey literature review. 
The qualitative questions posed can be effectively answered by a systematic literature review, as the movement is very young. 
Most if not all the literature on the subject can be reviewed, along with all the projects listed by the Open Source Hardware Association.

% We started with a seed of papers on this subject, and we checked the citations used in the seeds recursively. 

% Part of the methodology would also be to read through all the literature in the two FOSH journals and to record summaries, benchmarks, and the licences of the hardware they proposed.
% These two journals are the 
% \begin{enumerate}
%     \item \href{https://openhardware.metajnl.com/}{Journal of Open Hardware}
%     \item \href{https://www.sciencedirect.com/journal/hardwarex}{HardwareX}
% \end{enumerate}
\subsection{Search Strategy and Selection Criteria}

The publications of the FOSH journals were the initial pool of publications for our search. 

TODO add what we did

\subsection{Data Extraction and Analysis}

\subsection{Quality assessment}