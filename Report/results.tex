\documentclass[final-report.tex]{subfiles}

\begin{document}

The comprehensive results of our study can be accessed via the subsequent 
\href{https://github.com/aliraeisdanaei/FOSH_Lit_Review}{GitHub Repository}.

% TODO: FORMAT for Research Question A
FOSH is a new way of designing and building hardware that's become popular in many areas since it is open for innovation and can be democratically beneficial to the community. Based on the result of the systematic literature review, we can divide the perspective of FOSH in the following categories. electronics and computing, robotics and automation, and education and research.
\subsection{Education and Research}

% # aspect of education: electronic[4], auto vehicle design, industrial design, computer science, digital media design[3], K-12 student[92], control engineering education[98], robot[94], STEM education[]
% #kind of tool used: RISC-V based[18], Arduino[75],open-source openAirInterface Testbed project[62], Pantograph,3D printer, Hapkit[92], open-source robot platform(OSR)[94]
% #learning purpose, course: learning microprocessor, hardware based simulator[4], intergrate computer lab[18], microcontroller lab class[75], mobile communication network[62], robotic design[94], embedded system[]
% #newly invented devices for student to use: open-source power electronics didactic platform[61], open-source "ball-on-beam didactic device", attached to Arduino microcontroller [98], open-source robotic manipulation platform[148], development the front-end for the FPGA based platform[]
% #low-cost part: [3],[98], [148]


\subsection{Electronics and Computing}
Some of the most popular and successful hardware are open sourced, such as Risc-V and Arduino. 
These are free for modification, which is also commonly introduced and extended too various new products. 
For example, Mlakić1 et al. proposed a measurement device implemented based on Arduino. 
% TODO: REF 


\subsection{Robotics and Automation}
In general, robots are often created to either complete tasks that human is unable to, or to increase the productivity. 
VIKTOR III is an open source robot to improve the farming production quality by not only 
"empower[ing] individuals to grow their own food [but also] serve[ing] as platform for robotics education". 
This also falls into the category of the benefit of open source product can bring to the field of education. 
By improving on the current older invention, an open-sourced robot named FarmBot, VIKTOR III can be build only using 1/6 of its cost but with completely same feature. 
It also uses the state-of-art machine learning technology deep learning that provides a higher plant detection accuracy. 
The inventors also suggested that it can help astronauts harvest in space. By doing so, it provides a blueprint for future design and study purposes. 
There are many produces that share the same goal, such as the Yale OpenHand Project, which focus on improving the design process and produce various options for the researchers to adopt on.  
The inventor of WoodenHaptics also agrees with the importance of innovation, which is why they published the blueprint for this design in order to "Lowering the barriers to inspire experimentation". 
As a realm that is constantly been developed and updated by new technologies, they believe the ultimate future goal for any open source robot is "identifying willing end users who will put their own design modifications online, thereby allowing progress in the research community to move even faster". 
In fact, this concept of republishing the modified product is also widley and strongly agreed by many researchers, which leads to the invention of ROS (Robot Operating System) that allows people to communicate. 
The aspect of FOSH communities will be further discussed later. 

\end{document}