\documentclass{article}
\usepackage[a4paper, total={6in, 10in}]{geometry}

\usepackage{lipsum}

% \usepackage{float}
% \usepackage[]{graphicx}
% \graphicspath{{Images/}}
% \usepackage{hyperref}
% \usepackage{subfiles}

% \usepackage{biblatex}
% \addbibresource{ref.bib}

% \usepackage{booktabs}
% \usepackage{multicol}

% \usepackage{tabularx}

% \setlength\parindent{0pt}

% remove section numbering
% \setcounter{secnumdepth}{0}

\begin{document}
\pagenumbering{arabic}


\title{Project Description of a Systematic Literature Review of Free and Open-Source Hardware}
\author{Ali Raeisdanaei, Jingyue Zhang, Tiantian Lin, Ziqian Qiu }
\date{}
\maketitle

\section{Background}
The Free and Open-Source Software (FOSS) movement has made many strides over the years.
Free and Open-Source Hardware (FOSH) is an extension of the ethos of FOSS.
It aims to bring many of the free as in \textit{libre} values to the design and making of hardware to ensure privacy and ethical values interests of users.
FOSH has also had many improvements over the years, but it seems to have been slower than FOSH.

We propose to perform a systematic literature review of the research and industrial applications of FOSH to understand the following research questions:
\begin{enumerate}
    \item What are the challenges and drawbacks that FOSH is facing?
    \item What is the state-of-the art FOSH?
          \begin{enumerate}
              \item What are the performance of this best FOSH?
              \item How does it compare in terms of performance with the non-FOSH state-of-the-art?
              \item How does it compare in terms of other aspects with the non-FOSH state-of-the-art?
          \end{enumerate}

    \item Is FOSS more adapted to FOSH?
    \item What are the five and ten year predictions of the state-of-the-art FOSH?
\end{enumerate}

\section{Methods}
The main method we are planning to employ is systematic literature review.
We wish to provide a summary of the topics of the literature, and to connect related themes to each other.

% \nocite{*}
% \printbibliography

\end{document}