\documentclass{article}
\usepackage[a4paper, total={6in, 10in}]{geometry}

\usepackage{lipsum}

% \usepackage{float}
% \usepackage[]{graphicx}
% \graphicspath{{Images/}}
\usepackage{hyperref}
% \usepackage{subfiles}

\usepackage{biblatex}
\addbibresource{ref.bib}

% \usepackage{booktabs}
% \usepackage{multicol}

% \usepackage{tabularx}

% \setlength\parindent{0pt}

% remove section numbering
% \setcounter{secnumdepth}{0}

\begin{document}
\pagenumbering{arabic}


\title{Working Document of Project}
\author{Ali Raeisdanaei, Jingyue Zhang, Tiantian Lin, Ziqian Qiu }
\date{}
\maketitle

\section{Preface}
This is a drafting document created for the purpose of writing any ideas down.
This document is only a draft; therefore, it is rough and messy.
However, the full report should evolve out of this document.

\section{Research Questions}
Here are the research questions:

RQ1: What is the state-of-the art FOSH and what are the performance of this best FOSH? How does it compare in terms of performance and other aspects with the non-FOSH state-of-the-art? 

RQ2: What are the main challenges and drawbacks associated with the development, adoption, and sustainability of FOSH?

RQ3: How does non-FOSH compare with FOSH in terms of design and licsensing?

RQ4: To what extent is FOSS more adapted FOSH development, and what are the factors that contribute to any observed differences in their adaptability?

RQ5: What are the potential future developments, opportunities, and challenges that FOSH is facing, and what are the implications of these for the growth and sustainability of the FOSH movement?

\section{Introduction}
We need an introduction to what free and open source is. 
A history of this in the software realm and how this idea is extending to hardware is needed. 

We need to understand what are the values and principles of free and open source, and how they exactly relate to free and open source hardware (FOSH).
To this end we should consult the founder of it all Richard Stallman (RMS).
This should be a good source to read into
\textbf{Free Software, Free Society: Selected Essays of Richard M. Stallman}
\cite{b0_stallman}.

TODO write the summary of this section for the introduction citing Stallman.

An overview of hardware and its difficulties is needed as a good background information. 

Then, we also need to explain why a literature review is needed.
What exactly does our work add?

\section{Readings done by Ali}

I had vastly underestimated the amount of work in this field, and now I understand why a systematic literature review is needed.

\subsection{Defining success in open source hardware development projects: a survey of practitioners}
\cite{p1_def_succ}

\textbf{Very Important Paper for all of us to Read}

This paper lays out some very crucial starting point information.
It even has a section of literature review for us to start.

\subsubsection{Association}
There is an open source hardware association that tracks these hardware.
\href{https://www.oshwa.org}{Open Source Hardware Association}
This is a very crucial resource for us to dive into. 
The association gives out certification for projects that meet its requirements
\href{https://certification.oshwa.org/list.html}{https://certification.oshwa.org/list.html}

\subsubsection{Journal}
There have also been journals that have been dedicated to the FOSH.
\begin{enumerate}
    \item \href{https://openhardware.metajnl.com/}{Journal of Open Hardware}
    \item \href{https://www.sciencedirect.com/journal/hardwarex}{HardwareX}
\end{enumerate}

\subsubsection{Summary of Paper}

\section{Methodology of Literature Review}
We first identified a seed of papers, and recursively read the papers that were cited until saturation was reached. 

\nocite{*}
\printbibliography

\end{document}