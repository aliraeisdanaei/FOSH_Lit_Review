\documentclass{article}
\usepackage[a4paper, total={6in, 10in}]{geometry}

\usepackage{lipsum}

% \usepackage{float}
\usepackage[]{graphicx}
\graphicspath{{Images/}}
\usepackage{hyperref}
% \usepackage{subfiles}

\usepackage{biblatex}
\addbibresource{ref.bib}

% \usepackage{booktabs}
% \usepackage{multicol}

% \usepackage{tabularx}

% \setlength\parindent{0pt}

% remove section numbering
% \setcounter{secnumdepth}{0}

\begin{document}
\pagenumbering{arabic}


\title{Working Document of Project}
\author{Ali Raeisdanaei, Jingyue Zhang, Tiantian Lin, Ziqian Qiu }
\date{}
\maketitle

\section{Preface}
This is a drafting document created for the purpose of writing any ideas down.
This document is only a draft; therefore, it is rough and messy.
However, the full report should evolve out of this document.


\section{Introduction}

\subsection{Purpose of this Literature Review}
This project seeks to lay a ground build by systematic literature review for the layman and the practitioner to understand the free and open source hardware (FOSH).
It assumes no knowledge of the free and open source software (FOSS) movement, nor hardware.
Through explaining all of this and more from the available literature, the project seeks to answer what the state of FOSH by understanding its breadth and how it compares with nonFOSH.
\textbf{Then, we also need to explain why a literature review is needed.}
\textbf{What exactly does our work add?}

\subsection{Where is the Freedom?}
We have on our hands a revolution of technology. 
And it is not the endless march of \textit{forward} preached by exploitive companies.
This revolution is about \textbf{our} freedom to use technology for our learning, fun, and betterment.
We use technology; technology does not use us. 
The \textit{Free} in Free and Open Source is free as in freedom, as in liberty.
Freedom and liberty over our technology is the extension of our freedom and liberty over our personhood.
This revolution is championed by the Richard Stallman, computer scientist, philosopher, preacher, saint, and the founder of GNU and the Free Source Foundation \cite{b0_stallman}.
As the \textit{popular} song goes \textit{Remember Richard Stallman,
Who set your software free} \cite{song_GNUs_Not_Unix}.

\begin{figure}[h]
    \centering
    \includegraphics[width=\textwidth]{the_wise_gnu.jpg}
    \caption{Richard Matthew Stallman as depicted in lore}
    \label{fig:rms}
\end{figure}

The passion and revolution of this movement, for users to take back ownership of their technology, is incredibly powerful. 
No literature on the topic of open source anything can distance itself with this revolutionary zeal. 
While it has not been fully actualised, it has permeated many aspects of our lives over the years.
We have the rise of platforms such as GitHub and the widespread adoption of the Free Licenses preached by the Free and Open Source movement. 
Examples of its victories are the Linux kernel, Mozilla Firefox, \LaTeX, Vim, and other widespread software. 

To understand the definition of the \textit{free and open source} in free and open source hardware, one must turn to its founding essays. Here is a good working definition as quoted by the source itself:

\begin{quote}
    A program is free software if the program’s users have the four essential freedoms:

    \begin{itemize}
        \item The freedom to run the program as you wish, for any purpose (freedom 0).
        \item The freedom to study how the program works, and change it so it does your computing as you wish (freedom 1). 
            Access to the source code is a precondition for this.
        \item The freedom to redistribute copies so you can help your neighbor (freedom 2).
        \item The freedom to distribute copies of your modified versions to others (freedom 3). 
            By doing this you can give the whole community a chance to benefit from your changes. Access to the source code is a precondition for this.
    \end{itemize}
    A program is free software if it gives users adequately all of these freedoms. Otherwise, it is nonfree. 
    While we can distinguish various nonfree distribution schemes in terms of how far they fall short of being free, we consider them all equally unethical.
\end{quote} \cite{b0_stallman}.

This movement is about defining what technology means. 
It is rooted in the community and the liberty of sharing knowledge with one's neighbour. 
To be shackled in proprietary restraints on the intellectual property is to go against the definition of technology.

The main mechanism that this freedom is enforced is through licensing and mainly that of copyleft. 
Free software can be distributed as long as it remains free. 
This effectively prevents the software from becoming proprietary. 
This is what is known as copyleft, so called because it uses copyright law subversively \cite{b0_stallman}.
Under this category, there can be many licences that the Free Software Foundation considers to be free.
\href{https://www.gnu.org/licenses/license-list.html}{Free Licsences}
The most famous being the GNU Public License or GPL series.

\subsection{Where is the Hardware?}
An overview of hardware and its difficulties is needed as a good background information. 

The original Free and Open Source movement was about software not hardware.
There needs to then be an extension of the principles to Hardware.
Obviously the \textit{ware} in hardware is not the physical object, rather all the intellectual property required to design, build, and use the hardware. 
Similar to \cite{p1_def_succ}, these include all computer aided design (CAD) files, blueprints, bills of materials (BoMs), and so on. 
Therefore, the extension of FOSS to FOSH is a natural one, as they both concern the intellectual property themselves. 

\section{Research Questions}

\subsubsection{Freedom in OSH}
Open source is not the same thing as free and open source; in fact the \textit{free} derives the open source property \cite{b0_stallman}. 
To not do injustice to the ideals of the revolution, this project, will not drop the crucial \textit{'F'} in FOSH nor in writing nor in definition.
This theory and youthful, revolutionary spirit will be the guiding anchor of this review.

Many of the literature omits the crucial \textit{'F'} in FOSH.
In our literature review, we will determine the degree to which the original principle of freedom is lost not only in writing but in implementation.

RQ0: What are the types of licences that the OSH are using?

\subsubsection{What Hardware?}
\label{RQ1}
Hardware is a very broad term that encompasses many pieces of technology.
The Open Source Hardware Association lists [NUMBER] of hardware identified at [DATE].
The first research question of this literature review is to understand the breadth of free hardware.

RQ1: What are the available state-of-the-art FOSH in the different categories of hardware?

\subsubsection{Technical Specifications of FOSH}
As the stereotype goes, the devout follower of the free software ethos, will run an old-libre booted ThinkPad.
It will be slow, and out of date, but it is damn free. 
That is, while there has been many developments by champions of the movement, often, FOSS can be technically inferior to the proprietary solution. 
However, as Stallman puts it, 
\begin{quote}
it would have a social advantage, allowing users to cooperate, and an ethical advantage respecting the user's freedom 
\end{quote}
\cite{b0_stallman}.
The younger FOSH therefore, should be plagued with matters of technical inferiority a lot more than the FOSS. 
In this literature review, we wish to answer this hypothesis by asking the following questions:

RQ2: What is the state-of-the art FOSH? What are its technical specifications compared to nonFOSH? 

RQ3: What are the main challenges and drawbacks associated with the development, adoption, and sustainability of FOSH?

RQ4: What are the potential future developments, opportunities, and challenges that FOSH is facing, and what are the implications of these for the growth and sustainability of the FOSH movement?

\section{Readings done by Ali}

I had vastly underestimated the amount of work in this field, and now I understand why a systematic literature review is needed.

\subsection{Defining success in open source hardware development projects: a survey of practitioners}
\cite{p1_def_succ}

\textbf{Very Important Paper for all of us to Read}

This paper lays out some very crucial starting point information.
It even has a section of literature review for us to start.

\subsubsection{[Free and] Open Source Hardware Association}
There is a [free and] open source hardware association that tracks this hardware.
\href{https://www.oshwa.org}{Open Source Hardware Association}
This is a very crucial resource for us to dive into. 
The association gives out certification for projects that meet its requirements
\href{https://certification.oshwa.org/list.html}{https://certification.oshwa.org/list.html}

\subsubsection{[Free and] Open Source Hardware Journals}
There have also been journals that have been dedicated to the FOSH.
\begin{enumerate}
    \item \href{https://openhardware.metajnl.com/}{Journal of Open Hardware}
    \item \href{https://www.sciencedirect.com/journal/hardwarex}{HardwareX}
\end{enumerate}

\subsection{Prominent Projects to Dive into}
% \begin{enumerate}
    % \item https://reprap.org/wiki/RepRap
    % \item https://www.openacousticdevices.info/audiomoth
    % \item https://opentrons.com/
    % \item https://fossa.systems/
    % \item https://en.wikipedia.org/wiki/List_of_Electron_launches#2019
% \end{enumerate}

\subsubsection{Summary of Paper}
This paper studies the success of OSH projects.
It surveys practitioners of OSH to understand what is considered success at the project level both in terms of process and product. 
They corroborate their results with literature from OSS and nonOSH.
The number of participants was 30, and the survey was done in an open manner.

They identified the following success criterion:
\begin{enumerate}
    \item Create value
        This is the positive outcome or importance of the project. 
    \item Create high-quality outputs
    \item Have effective processes
\end{enumerate}

\section{Methodology of Literature Review}
We first identified a seed of papers, and recursively read the papers that were cited.

Read all the papers ever published by HardwareX to date of version 14. 
Read all the papers ever published by Journal of Open Hardware to date.

\section{Results}

\subsection{Freedom}
The Open Source Hardware Association lists the following as the main criterion for its definition:
\begin{quote}
Open source hardware is hardware whose design is made publicly available so that anyone can study, modify, distribute, make, and sell the design or hardware based on that design 
\end{quote}
\cite{OSH_association_def}.

Without clearly stating freedom anywhere, this definition is compatible with that of Richard Stallman's in \cite{b0_stallman}.

To understand the scope of the agreement of all the projects listed by the Open Source Hardware Association, we 
TODO FILL IN WHAT NEEDS TO GO IN HERE

\nocite{*}
\printbibliography

\end{document}